\documentclass[10pt,a4paper]{article}
\usepackage[utf8]{inputenc}
\usepackage[spanish]{babel}
\usepackage{amsmath}
\usepackage{amsfonts}
\usepackage{amssymb}
\usepackage{graphicx}
\usepackage{pgfplots}
\pgfplotsset{compat=1.8}
\usepgfplotslibrary{statistics}
\usepackage[left=1.00cm, right=1.00cm, top=1.00cm, bottom=2.00cm]{geometry}
\begin{document}
$\dagger$	Diga cual de las siguientes expresiones es falsa cuando nos referimos al ANOVA:
	\begin{itemize}
		\item a) Los tratamientos son una fuente de variación.
		\item b) Los grados de libertad de la variación del error es el numero de unidades experimentales menos uno.
		\item c) Las poblaciones tienen iguales varianzas.
		\item d) Se calcula el cociente de dos varianzas muestrales y se lo compara con un valor de F.
		\item e) Un cuadrado medio es una varianza estimada.
		\item f) La región critica para la regla de decisión es unilateral derecha.
	\end{itemize}

$\dagger$ Si el suceso B esta incluido en A o dicho de otra manera, está dentro de A, señale la respuesta correcta:
\begin{itemize}
	\item a) $P(A)<P(B)$
	\item b) $P(A\cap B) = P(B)$
	\item c) $P(A\cap B) = 0$
	\item d) Ninguna de las respuestas es correcta 
\end{itemize}

$\dagger$ ¿Qué medidas descriptivas (tendencia central, dispersión, posición) se pueden leer
\begin{itemize}
	\item a) en el gráfico de caja y bigote (box-plot)?
	\item b) en el histograma?
	\item c) en el polígono de frecuencias absolutas?
	\item d) en la ojiva (de frecuencias acumuladas)?
\end{itemize}

$\dagger$ Indique la o las expresiones correctas. En un histograma de frecuencias absolutas, lo que representa a la frecuencia correspondiente a un intervalo es:
\begin{itemize}
	\item a) la altura del rectángulo.
	\item b) la altura correspondiente a la marca de clase.
	\item c) el área del rectángulo.
	\item d) la altura correspondiente a la marca de clase.
	\item e) ninguna de las anteriores.
\end{itemize}


$\dagger$ ¿Cuando dos sucesos son mutuamente excluyentes?

$\dagger$ Defina o explique claramente que es:
\begin{itemize}
	\item 1) Cuartilo
	\item 2) Modo
	\item 3) Función de cuantía
\end{itemize}

$\dagger$ Indique si cada afirmación es correcta o incorrecta. En los casos en que la afirmación sea incorrecta justifique su respuesta.
\begin{itemize}
	\item 1) $R^2$ representa la fracción de la variación total en Y que no esta explicada.
	\item 2) Un gran fabricante de automóviles ha tenido que retirar varios modelos de su linea 1993 debido a problemas de control de calidad que no fueron descubiertos con los procedimientos finales de inspección aleatoria. Este es un ejemplo de Error de tipo II.
\end{itemize}

$\dagger$ Una cierta población distribuida normalmente tiene una desviación estándar conocida de 1.0 ¿ Cual es el ancho total de un intervalo de confianza de 95\% para la media de la población?
\begin{itemize}
	\item a) 1.96
	\item b) 0.98
	\item c) 3.92
	\item d) No se puede determinar de la información dada
\end{itemize}

$\dagger$ Si $P(A\cap B) = P(A)P(B/A)$ es porque:
\begin{itemize}
	\item a) A y B son mutuamente excluyentes? Si/No
	\item b) A y B son sucesos dependientes? Si/No
	\item c) A y B son sucesos independientes? Si/No
\end{itemize}

$\dagger$ Suponga que en un corral hay doce machos y ocho hembras y se seleccionan al azar y se separan para ser tratados 5 de los animales. La siguiente expresión, da la probabilidad de obtener:
\begin{equation}
	\frac{\genfrac(){0pt}{2}{12}{2}\genfrac(){0pt}{2}{8}{3}}{\genfrac(){0pt}{2}{20}{5}}
\end{equation}

\begin{itemize}
	\item a) Al menos dos machos en los cinco ensayos
	\item b) Exactamente dos machos en cinco ensayos
	\item c) Exactamente dos machos en seis ensayos
	\item d) Exactamente dos machos en cinco ensayos
	\item e) Exactamente tres hembras en cuatro ensayos
	\item f) Ninguna de las anteriores. ¿Cual?
\end{itemize}

$\dagger$ Indique si cada una de las expresiones es correcta para expresarse como conclusión o interpretación, después de la construcción de un intervalo de confianza para un promedio población. (Al construir el intervalo empleando un coeficiente de confianza de 0.95, se encontró el limite inferior 10 y el limite superior 25)

\begin{itemize}
	\item a) $P(10<\mu <25) = 0.95$ SI / NO, porque...
	\item b) $P(\mu=17.5)=0.95$ SI / NO, porque...
	\item c) $0<\mu< 25$ con un coeficiente de confianza 0.95 SI / NO, porque
\end{itemize}

$\dagger$ Explique que significa estadísticamente "$\beta$":
\begin{itemize}
	\item a) en un problema de regresión lineal simple
	\item b) en un problema de Dócima o prueba de hipótesis
\end{itemize}

$\dagger$ Señale con una cruz cuales son las características de una distribución Hipergeométrica:
\begin{itemize}
	\item a) Variable Continua
	\item b) Variable discreta
	\item c) Tiene dos resultados posibles
	\item d) Tiene dos resultados posibles (mutuamente excluyentos)
	\item e) Tiene mas de dos resultados posibles
	\item f) El experimento básico se repite un numero finito de veces
	\item g) El experimento básico se repite hasta el éxito
	\item h) Las probabilidades se mantienen constantes
	\item i) Las probabilidades no se mantienen constantes
\end{itemize}

$\dagger$ Una variable de Poisson se caracteriza porque:
\begin{itemize}
	\item a) Es muy pequeña la probabilidad de un suceso elemental (o simple)
	\item b) Los sucesos elementales son independientes entre si en cada experimento
	\item c) Cuenta el que se repita cierto numero de veces un suceso elemental a lo largo de un continuo
	\item d) Se debe verificar todo lo anterior simultáneamente
	\item e) Cuenta en que momento ocurre por primera vez un suceso elemental
\end{itemize}

$\dagger$ Dados dos sucesos A y B el teorema del producto se expresa:
\begin{equation}
	P(A o B) = P(A \cup B) = P(A)xP(B)
\end{equation}
\begin{itemize}
	\item a) cuando A y B son independientes
	\item b) siempre
	\item c) nunca
	\item d) cuando P(A) y P(B) son independientes
	\item e) cuando A y B son mutuamente excluyentes
\end{itemize}

$\dagger$ ¿Cual de los siguientes es el primer paso para calcular la mediana de un conjunto pequeño de datos?
\begin{itemize}
	\item a) Calcular la mediana de orden
	\item b) Ordenar los datos
	\item c) Determinar las frecuencias relativas de los valores de los datos
	\item d) Sumar los valores de la variable y dividir por la mitad
	\item e) Encontrar el valor que deja la mitad de los valores a cada lado de el
	\item f) Ninguno de los anteriores
\end{itemize}

$\dagger$ La diferencia entre una variable aleatoria con distribución binomial y una distribución hipergeométrica:
\begin{itemize}
	\item a) a la cantidad de observaciones
	\item b) a que en las dos hay dos resultados posibles
	\item c) a la aleatoriedad
	\item d) a ninguno de los 3.
\end{itemize}

$\dagger$ Si un estimador es insesgado la esperanza del estimador coincide con el valor del parámetro.
\begin{itemize}
	\item a) Siempre
	\item b) A veces
	\item c) Nunca
	\item d) Ninguna de las anteriores 
\end{itemize}

$\dagger$ La probabilidad de no rechazar una hipótesis nula cuando es cierta es:
\begin{itemize}
	\item a) el nivel de confianza
	\item b) el tamaño de la región critica
	\item c) el nivel de sesgo
	\item d) ninguna de las tres
\end{itemize}

$\dagger$ Razonar para cuales de los siguientes problemas la distribución binomial es un modelo adecuado:
\begin{itemize}
	\item a) Determinación de la probabilidad de que un agente de ventas lleve a cabo 2 ventas en 5 entrevistas independientes si la probabilidad es 0.25 de que el agente lleve a cabo una venta en una entrevista determinada
	\item b) Determinación de la probabilidad de que no mas de 1 de 10 artículos producidos por una maquina sea defectuoso cuando los artículos se seleccionan a través del tiempo y se sabe que la proporción de defectuosos aumenta con el desgaste de la maquina con el tiempo.
\end{itemize}

$\dagger$ ¿Cuando dos sucesos son mutuamente excluyentes?

$\dagger$ ¿Cuando dos sucesos son probabilísticamente independientes?

$\dagger$ En cada uno de los casos siguientes se pide:
\begin{itemize}
	\item a) Reconocer de que aplicación de "usos de chi" se trata.
	\item b) escribir en palabras la hipótesis nula y alternativa.
\end{itemize}

Un científico social selecciono una muestra de 140 personas y las clasifico de acuerdo a su nivel de ingreso y si jugaron o no al Bingo durante el ultimo mes. La tabla de contingencia obtenida fue:
\begin{center}
	\begin{tabular}{|c|c|c|c|c|}
		\hline 
		& \multicolumn{4}{c|}{Ingreso} \\ 
		\hline 
		jugó? & bajo & medio & alto & TOTAL \\ 
		\hline 
		jugó & 46 & 28 & 21 & 95 \\ 
		\hline 
		no jugó & 14 & 12 & 19 & 45 \\ 
		\hline 
		TOTAL & 60 & 40 & 40 & 140 \\ 
		\hline 
	\end{tabular} 
\end{center}

Un científico social selecciono muestras de personas de diferente nivel de ingreso y las clasifico de acuerdo a si jugaron o no al bingo durante el ultimo mes. La tabla de contingencia obtenida fue:

\begin{center}
	\begin{tabular}{|c|c|c|c|c|}
		\hline 
		& \multicolumn{4}{c|}{Ingreso} \\ 
		\hline 
		jugó? & bajo & medio & alto & TOTAL \\ 
		\hline 
		jugó & 46 & 28 & 21 & 95 \\ 
		\hline 
		no jugó & 14 & 12 & 19 & 45 \\ 
		\hline 
		TOTAL & 60 & 40 & 40 & 140 \\ 
		\hline 
	\end{tabular} 
\end{center}

$\dagger$ Si la variable aleatoria discreta toma solo valores ${1,2,5,8,9}$ la función de distribución acumulada, $F(x)$:
\begin{itemize}
	\item a) Sólo toma valores mayores o iguales que cero y menores o iguales que uno
	\item b) Puede calcularse a partir de la función de masa de probabilidad
	\item c) Es posible calcularla para el valor de la variable 5,39
	\item d) Todas las anteriores
\end{itemize}

$\dagger$ Definir error de tipo I y error de tipo II, ¿cuando se los utiliza?, ¿como se los designa?

$\dagger$ Diferencias entre homogeneidad e independencia en cuanto a su valor esperado, la región critica y muestreo.

$\dagger$ ¿Qué condiciones se deben cumplir para que haya una función de cuantía?


$\dagger$ Nombrar los supuestos de regresión simple

$\dagger$ Nombrar las condiciones para que se cumpla ANOVA

$\dagger$ A partir de una tabla de ANOVA, ¿Qué hipótesis nula y alternativa puede probar?
\begin{itemize}
	\item a) $H_0 : \beta = 0.833$ con $H_1 : \beta \neq 0.833$?
	\item b) $H_0 : \beta = 0$ con $H_1 : \beta > 0$?
	\item c) $H_0 : \beta = 0$ con $H_1 : \beta \neq 0$?
	\item d) Otra ...
\end{itemize}

$\dagger$ Responda VERDADERO si la oración que se le presenta es SIEMPRE verdadera. Si esa oración no es siempre verdadera reemplace la palabra en negrita por palabra/s que hagan siempre verdadera la afirmación.

\begin{itemize}
	\item a) La \textbf{media} de los datos siempre divide a estos en mitades, la mitad mas chicos y la mitad mas grandes que ella
	\item b) La suma de los cuadrados de las desviaciones de la variable respecto de la media $\sum (x_i-x)^2$, \textbf{a veces es negativa}
\end{itemize}

$\dagger$ Coloque verdadero (V) o falso (F). Justifique cuando sea falso.
\begin{itemize}
	\item a) Al calcular la mediana se toman en consideración todos los valores de la variable
	\item b) En un box plot, la linea central de la caja es la media aritmética
\end{itemize}

$\dagger$ Se tienen los cinco datos necesarios para construir un diagrama de caja y bigote.
\begin{equation*}
	29 \hspace{5mm} 8 \hspace{5mm} 2 \hspace{5mm} 6 \hspace{5mm} 15
\end{equation*}

\begin{itemize}
	\item a) La linea central de la caja es: La mediana, la media aritmética, el modo, la varianza?
	\item b) El borde derecho de la caja es el: primer cuartilo (Q1), segundo cuartilo (Q2), tercer cuartilo (Q3)?
	\item c) Ubique los cinco datos en el diagrama.	
\end{itemize}

\begin{center}
	\begin{tikzpicture}
	\begin{axis}
	[
	ytick={1,2,3},
	yticklabels={Index 0},
	]
	\addplot+[
	boxplot prepared={
		median=0.6,
		upper quartile=1.2,
		lower quartile=0.4,
		upper whisker=1.5,
		lower whisker=0.2
	},
	] coordinates {};
	\end{axis}
	\end{tikzpicture}
\end{center}

$\dagger$ Se ha medido la longitud de los 20 ejemplares capturados en ultimo torneo de pesca, obteniéndose:
\begin{center}
	\begin{tabular}{|c|c|c|c|c|c|c|c|c|c|}
		\hline 
		11 & 14.6 & 8.4 & 7.4 & 2.8 & 15.5 & 9.9 & 12.3 & 14.2 & 2.3 \\ 
		\hline 
		10.7 & 4.6 & 9.0 & 5.6 & 7.7 & 10.1 & 11.8 & 12.7 & 10.4 & 14.8 \\ 
		\hline 
	\end{tabular}
\end{center} 

\begin{itemize}
	\item la mediana de los datos es (2.3-10.7)/2 = 6.5
	\subitem a) SI, porque ...
	\subitem b) NO, porque ...
	\subitem c) NO PUEDO CALCULARLA, porque ...
	\item la media aritmética de los datos podría ser 15.6
	\subitem a) SI, porque ...
	\subitem b) NO, porque ...
	\subitem c) NO PUEDO CALCULARLA, porque ...
\end{itemize}

$\dagger$  Analice las cajas:

\begin{center}
	\begin{tikzpicture}
	\begin{axis}
	[
	ytick={1,2},
	yticklabels={Index 0, Index 1},
	]
	\addplot+[
	boxplot prepared={
		median=0.6,
		upper quartile=1.2,
		lower quartile=0.4,
		upper whisker=1.7,
		lower whisker=0.2
	},
	] coordinates {};
	\addplot+[
	boxplot prepared={
		median=0.9,
		upper quartile=1.2,
		lower quartile=0.4,
		upper whisker=1.7,
		lower whisker=0.2
	},
	] coordinates {};
	\end{axis}
	\end{tikzpicture}
\end{center}

Es cierto que ahora SI, en la caja del segundo diagrama, hay un 25\%  a cada lado de la mediana y en el primero NO? Diga SI o NO y justifique.\\

$\dagger$ Seleccione LA respuesta correcta. Señale con una cruz X.
\begin{itemize}
	\renewcommand{\labelitemi}{\raisebox{-.25\height}{\huge$\square$}}
	\item En una distribución Hipergeométrica, los dos resultados posibles no son mutuamente excluyentes porque no son independientes.
	\item La diferencia fundamental a tener en cuenta para saber si se emplea un modelo Binomial o uno Hipergeométrico es analizar si las repeticiones son o no son independientes.
	\item La característica común de las tres distribuciones discretas (binomial, hipergeométrica, poisson) es que, además de tratarse de variables discretas, el numero de repeticiones del experimento básico es conocido de antemano.
\end{itemize}

$\dagger$  Marque las/las respuestas correctas:

\begin{itemize}
	\renewcommand{\labelitemi}{\raisebox{-.25\height}{\huge$\square$}}
	\item En una función de distribución para variable discreta la altura de la gráfica en cada punto de la variable representa la probabilidad en ese punto.
	\item En una función de distribución para variable continua la altura de la gráfica en cada punto de la variable representa la probabilidad acumulada hasta ese valor de la variable.
	\item En una función de densidad la altura de la gráfica en cada punto de la variable representa la probabilidad acumulada hasta ese valor de la variable.
\end{itemize}

$\dagger$ ¿Cual es la variable en un esquema o modelo binomial?
\begin{itemize}
	\renewcommand{\labelitemi}{\raisebox{-.25\height}{\huge$\square$}}
	\item El numero de repeticiones del experimento básico
	\item El porcentaje de éxitos
	\item El numero de éxitos en las n repeticiones del experimento básico.
	\item El numero de las repeticiones hasta el primer éxito
	\item Ninguna de las anteriores
\end{itemize}
$\dagger$ Sea la siguiente tabla:
\begin{center}
	\begin{tabular}{|c|c|c|c|c|}
		\hline 
		variable & 0 & 10 & 100 & 500 \\ 
		\hline 
		probabilidad & 0.45 & ... & 0.4 & 0.05 \\ 
		\hline 
	\end{tabular}
\end{center} 
Complete la tabla para que sea función de probabilidad.
\\
$\dagger$ Diga cual de las siguientes afirmaciones es falsa:
\begin{itemize}
	\renewcommand{\labelitemi}{\raisebox{-.25\height}{\huge$\square$}}
	\item Si de una población en la cual cada repetición del experimento básico tiene dos resultados mutuamente excluyentes se extrae una muestra de tamaño 10, con reposición, la variable numero de éxitos se distribuye segun una distribución hipergeométrica.
	\item Si de una población se extrae una muestra de tamaño 10, sin reposición, la variable numero de éxitos se distribuye segun una distribución hipergeométrica, si en cada repetición del experimento básico hay dos resultados mutuamente excluyentes.
	\item Si en un experimento es correcto emplear un modelo binomial cuando las extracciones se hacen sin reposición, es porque se conoce el porcentaje de éxitos en la población o porque la población es lo suficientemente grande.
\end{itemize}

$\dagger$ ¿Cual es la diferencia entre una distribución binomial y una hipergeométrica?
\begin{itemize}
	\renewcommand{\labelitemi}{\raisebox{-.25\height}{\huge$\square$}}
	\item La variable en la binomial es discreta y en la hipergeométrica es continua.
	\item En la distribución binomial la probabilidad de éxito es constante en las diferentes repeticiones y en la hipergeométrica cambia en cada repetición
	\item En la distribución hipergeométrica la probabilidad de éxito es constante en las diferentes repeticiones y en la binomial cambia en cada repetición
	\item No hay diferencias entre las dos distribuciones
\end{itemize}

$\dagger$  De acuerdo con el Teorema Central del Limite: ¿Podemos esperar que la media de la distribución muestral de medias se aproxime a la media de la población?

$\dagger$ Señale la o las expresión/es correcta/s
\begin{itemize}
	\renewcommand{\labelitemi}{\raisebox{-.25\height}{\huge$\square$}}
	\item El estimador puntual del parámetro esta encerrado en el intervalo de confianza
	\item El parametro que se estima esta encerrado en el intervalo de confianza
	\item Ninguna de las anteriores.
\end{itemize}

$\dagger$ Al construir el intervalo para la media población empleando un coeficiente de confianza de 0.95, se entro que el limite inferior es 10 y el limite superior es 25. ¿Es correcto decir que: Se tiene una probabilidad de 0.95 de que esos números contendrían al verdadero valor del parámetro?

$\dagger$ En el siguiente conjunto de datos vericar:

\begin{center}
	\begin{tabular}{|c|c|c|c|c|c|}
		\hline 
		4 & 9 & 3 & 6 & 2 & 7 \\ 
		\hline 
	\end{tabular} 
\end{center}

\begin{itemize}
	\item $E(X+c) = E(x)+c$
	\item $E(cX) = cE(x)$
	\item $\sigma^2(X+c)=\sigma^2(X)$
	\item $\sigma^2(cX)=c^2E(X)$
\end{itemize}

$\dagger$ Si la variable aleatoria discreta toma sólo valores ${1,2,5,8,9}$, la función de distribución acumulada, F(x)
\begin{itemize}
	\item a) Solo toma valores mayores o iguales que cero, y menores o iguales que uno.
	\item b) Puede calcularse a partir de la función de masa de probabilidad f(x)
	\item c) Es posible calcularla para el valor de la variable 5,39.
	\item d) Todas las anteriores
\end{itemize}

$\dagger$  Señale la o las expresión/es correcta/s
\begin{itemize}
	\renewcommand{\labelitemi}{\raisebox{-.25\height}{\huge$\square$}}
	\item El estimador puntual del parámetro está encerrado en el intervalo de confianza
	\item El parámetro que se estima está encerrado en el intervalo de confianza
	\item Ninguna de las anteriores
\end{itemize}

$\dagger$  El gerente de una empresa de transportes recibe un lote de paquetes a transportar y desea comparar la variabilidad relativa entre el peso de los paquetes y el volumen que ocupan ¿que debe de utilizar? Señale la/s respuesta correcta.

\begin{itemize}
	\item Las desviaciones típicas.
	\item Los rangos.
	\item Los coeficientes de variación.
	\item La diferencia de las medias.
	\item La diferencia de las varianza.
\end{itemize}

$\dagger$  Indique si las siguientes expresiones son verdaderas o falsas.
\begin{itemize}
	\renewcommand{\labelitemi}{\raisebox{-.25\height}{\huge$\square$}}
	\item En una tabla de distribución de frecuencias se utilizan intervalos mutuamente excluyentes
	\item El valor de la mediana se encuentra afectado por la presencia de valores extremos
	\item En un conjunto de datos el valor calculado de la mediana siempre se encuentra entre los valores de la media aritmética y la moda.
	\item La sumatoria de las diferencias entre cada uno de los valores con respecto a la media aritmética siempre debe ser igual a cero.
\end{itemize}

$\dagger$  Señale cual de las siguientes afirmaciones es falsa:
\begin{itemize}
	\item La media aritmética es siempre el centro de gravedad de la distribución.
	\item En una distribución continua simétrica, media y mediana coinciden.
	\item La media aritmética cambia cuando cambia algún dato.
	\item La mediana no siempre cambia cuando lo hace algún dato.
	\item En las distribuciones continuas simétricas todas las medidas de centralización coinciden.
\end{itemize}

$\dagger$   Seleccione la respuesta correcta. Cuando a todos los datos de una muestra se les suma una constante.
\begin{itemize}
	\item a) La media no varia
	\item b) La media queda incrementada en esa constante
	\item c) El desvío estándar no varia
	\item d) El desvío estándar queda aumentado en esa constante
\end{itemize}
\newpage
$\dagger$  Entre el primero y el tercer cuartilo se encuentra:
\begin{itemize}
	\renewcommand{\labelitemi}{\raisebox{-.25\height}{\huge$\square$}}
	\item El 100\%
	\item La mitad de la distribución
	\item El 25\% de la distribución
	\item El 10\% de la distribución
	\item Ninguna de las anteriores 
\end{itemize}

$\dagger$  Indique si las siguientes afirmaciones son verdaderas o falsas
\begin{itemize}
	\item a) La función de probabilidad f(x) de una variable aleatoria continua X, siempre y sin restricciones, asume valores iguales o mayores a cero.
	\item b) Si de una población dicotomica se extrae una muestra con reemplazamiento, la distribución de probabilidad que resulta es la hipergeométrica
	\item c) Cuando se utiliza la distribución binomial en un estudio de opinión se esta suponiendo que la población es grande.
	\item d) La distribución hipergeométrica se puede aproximar por un distribución binomial cuando la muestra se extrae de una población grande.
\end{itemize}

$\dagger$  Marque la respuesta correcta. Comparar la media con la mediana de un conjunto de datos te da una idea de lo esparcidos que se encuentran los valores del conjunto de datos.
\begin{itemize}
	\item La media y la mediana tienen que coincidir para saber esto
	\item Si la media es mayor que la mediana los datos están mal
	\item Si la media es menor que la mediana los datos están mal
	\item Cuando la media y la mediana distan mucho los datos están muy desperdigados
\end{itemize}

$\dagger$  Sea B un suceso contenido en el suceso A, $B\in A$, ¿Que opción es correcta?
\begin{itemize}
	\item $P(B)<P(A)$
	\item $P(B\cup A)=P(B)+P(A)$
	\item $P(A\cap B)=0$
	\item Ninguna de las anteriores
\end{itemize}

$\dagger$  Si F es una función de distribución de una variable aleatoria continua, entonces cumple:
\begin{itemize}
	\item F(x) está acotado entre 0 y 1.
	\item F es estrictamente creciente
	\item $\int_{0}^{+\infty}=1$
	\item ninguna de las anteriores
\end{itemize}

$\dagger$  Mencione las condiciones de la distribución Binomial e indique ¿Bajo que condiciones la distribución Binomial puede ser aproximada por la distribución de Poisson?

$\dagger$  Indique si las siguientes afirmaciones son Verdaderas o Falsas. De acuerdo al teorema central del limite:
\begin{itemize}
	\item a) Cuando se disminuye el tamaño de la muestra, se disminuye el error estándar de la media (la desviación estándar de la media muestral)
	\item b) Podemos esperar que la media de la distribución maestral de medias se aproxime a la media de la población
	\item c) Habrá mas dispersión en la distribución de las medias que en la población
\end{itemize}

$\dagger$  Explique los posibles errores a cometer en la decisión de una prueba de hipótesis. Su relación con el nivel de significación.

$\dagger$  Enuncie los supuestos teóricos para el modelo de regresión lineal simple

$\dagger$  Defina coeficiente de correlación y coeficiente de determinación. Explique en que caso los utiliza y como interpreta a cada uno de ellos

$\dagger$  En un estudio de análisis de la Varianza el investigador forma cinco grupos y toma cuatro observaciones en cada uno de ellos. ¿Cuantas unidades experimentales hay?¿ Cuantos tratamientos? Escriba las hipótesis en términos de los efectos y en términos de los promedios 



\end{document}
