\documentclass[10pt,a4paper]{article}
\usepackage[utf8]{inputenc}
\usepackage[spanish]{babel}
\usepackage{amsmath}
\usepackage{amsfonts}
\usepackage{amssymb}
\usepackage{graphicx}
\usepackage[left=1.00cm, right=1.00cm, top=1.00cm, bottom=2.00cm]{geometry}
\begin{document}
$\dagger$	Diga cual de las siguientes expresiones es falsa cuando nos referimos al ANOVA:
	\begin{itemize}
		\item a) Los tratamientos son una fuente de variacion.
		\item b) Los grados de libertad de la variacion del error es el numero de unidades experimentales menos uno.
		\item c) Las poblaciones tienen iguales varianzas.
		\item d) Se calcula el cociente de dos varianzas muestrales y se lo compara con un valor de F.
		\item e) Un cuadrado medio es una varianza estimada.
		\item f) La región critica para la regla de decision es unilateral derecha.
	\end{itemize}

$\dagger$ Si el suceso B esta incluido en A o dicho de otra manera, está dentro de A, señale la respuesta correcta:
\begin{itemize}
	\item a) $P(A)<P(B)$
	\item b) $P(A\cap B) = P(B)$
	\item c) $P(A\cap B) = 0$
	\item d) Ninguna de las respuestas es correcta 
\end{itemize}

$\dagger$ ¿Qué medidas descriptivas (tendencia central, dispersion, posicion) se pueden leer
\begin{itemize}
	\item a) en el gráfico de caja y bigote (box-plot)?
	\item b) en el histograma?
	\item c) en el poligono de frecuencias absolutas?
	\item d) en la ojiva (de frecuencias acumuladas)?
\end{itemize}

$\dagger$ Indique la o las expresiones correctas. En un histograma de frecuencias absolutas, lo que representa a la frecuencia correspondiente a un intervalo es:
\begin{itemize}
	\item a) la altura del rectangulo.
	\item b) la altura correspondiente a la marca de clase.
	\item c) el area del rectangulo.
	\item d) la altura correspondiente a la marca de clase.
	\item e) ninguna de las anteriores.
\end{itemize}


$\dagger$ ¿Cuando dos sucesos son mutuamente excluyentes?

$\dagger$ Defina o explique claramente que es:
\begin{itemize}
	\item 1) Cuartilo
	\item 2) Modo
	\item 3) Función de cuantía
\end{itemize}

$\dagger$ Indique si cada afirmacion es correcta o incorrecta. En los casos en que la afirmacion sea incorrecta justifique su respuesta.
\begin{itemize}
	\item 1) $R^2$ representa la fraccion de la variacion total en Y que no esta explicada.
	\item 2) Un gran fabricante de automoviles ha tenido que retirar varios modelos de su linea 1993 debido a problemas de control de calidad que no fueron descubiertos con los procedimientos finales de inspeccion aleatoria. Este es un ejemplo de Error de tipo II.
\end{itemize}

$\dagger$ Una cierta poblacion distribuida normalmente tiene una desviacion estandar conocida de 1.0 ¿ Cual es el ancho total de un intervalo de confianza de 95\% para la media de la poblacion?
\begin{itemize}
	\item a) 1.96
	\item b) 0.98
	\item c) 3.92
	\item d) No se puede determinar de la informacion dada
\end{itemize}

$\dagger$ Si $P(A\cap B) = P(A)P(B/A)$ es porque:
\begin{itemize}
	\item a) A y B son mutuamente excluyentes? Si/No
	\item b) A y B son sucesos dependientes? Si/No
	\item c) A y B son sucesos independientes? Si/No
\end{itemize}

$\dagger$ Suponga que en un corral hay doce machos y ocho hembras y se seleccionan al azar y se separan para ser tratados 5 de los animales. La siguiente expresion, da la probabilidad de obtener:
\begin{equation}
	\frac{\genfrac(){0pt}{2}{12}{2}\genfrac(){0pt}{2}{8}{3}}{\genfrac(){0pt}{2}{20}{5}}
\end{equation}

\begin{itemize}
	\item a) Al menos dos machos en los cinco ensayos
	\item b) Exactamente dos machos en cinco ensayos
	\item c) Exactamente dos machos en seis ensayos
	\item d) Exactamente dos machos en cinco ensayos
	\item e) Exactamente tres hembras en cuatro ensayos
	\item f) Ninguna de las anteriores. ¿Cual?
\end{itemize}

$\dagger$ Indique si cada una de las expresiones es correcta para expresarse como conclusion o interpretacion, despues de la construccion de un intervalo de confianza para un promedio poblacional. (Al construir el intervalo empleando un coeficiente de confianza de 0.95, se encontró el limite inferior 10 y el limite superior 25)

\begin{itemize}
	\item a) $P(10<\mu <25) = 0.95$ SI / NO, porque...
	\item b) $P(\mu=17.5)=0.95$ SI / NO, porque...
	\item c) $0<\mu< 25$ con un coeficiente de confianza 0.95 SI / NO, porque
\end{itemize}

$\dagger$ Explique que significa estadisticamente "$\beta$":
\begin{itemize}
	\item a) en un problema de regresion lineal simple
	\item b) en un problema de Dócima o prueba de hipótesis
\end{itemize}

$\dagger$ Señale con una cruz cuales son las caracteristicas de una distribucion Hipergeometrica:
\begin{itemize}
	\item a) Variable Continua
	\item b) Variable discreta
	\item c) Tiene dos resultados posibles
	\item d) Tiene dos resultados posibles (mutuamente excluyentos)
	\item e) Tiene mas de dos resultados posibles
	\item f) El experimento basico se repite un numero finito de veces
	\item g) El experiment basico se repite hasta el exito
	\item h) Las probabilidades se mantienen constantes
	\item i) Las probabilidades no se mantienen constantes
\end{itemize}

$\dagger$ Una variable de Poisson se caracteriza porque:
\begin{itemize}
	\item a) Es muy pequeña la probabilidad de un suceso elemental (o simple)
	\item b) Los sucesos elementales son independientes entre si en cada experimento
	\item c) Cuenta el que se repita cierto numero de veces un suceso elemental a lo largo de un continuo
	\item d) Se debe verificar todo lo anterior simultaneamente
	\item e) Cuenta en que momento ocurre por primera vez un suceso elemental
\end{itemize}

$\dagger$ Dados dos sucesos A y B el teorema del producto se expresa:
\begin{equation}
	P(A o B) = P(A \cup B) = P(A)xP(B)
\end{equation}
\begin{itemize}
	\item a) cuando A y B son independientes
	\item b) siempre
	\item c) nunca
	\item d) cuando P(A) y P(B) son independientes
	\item e) cuando A y B son mutuamente excluyentes
\end{itemize}

$\dagger$ ¿Cual de los siguientes es el primer paso para calcular la mediana de un conjunto pequeño de datos?
\begin{itemize}
	\item a) Calcular la mediana de orden
	\item b) Ordenar los datos
	\item c) Determinar las frecuencias relativas de los valores de los datos
	\item d) Sumar los valores de la variable y dividir por la mitad
	\item e) Encontrar el valor que deja la mitad de los valores a cada lado de el
	\item f) Ninguno de los anteriores
\end{itemize}

$\dagger$ La diferencia entre una variable aleatoria con distribucion binomial y una distribucion hipergeometrica:
\begin{itemize}
	\item a) a la cantidad de observaciones
	\item b) a que en las dos hay dos resultados posibles
	\item c) a la aleatoriedad
	\item d) a ninguno de los 3.
\end{itemize}

$\dagger$ Si un estimador es insesgado la esperanza del estimador coincide con el valor del parámetro.
\begin{itemize}
	\item a) Siempre
	\item b) A veces
	\item c) Nunca
	\item d) Ninguna de las anteriores 
\end{itemize}

$\dagger$ La probabilidad de no rechazar una hipotesis nula cuando es cierta es:
\begin{itemize}
	\item a) el nivel de confianza
	\item b) el tamaño de la region critica
	\item c) el nivel de sesgo
	\item d) ninguna de las tres
\end{itemize}

$\dagger$ Razonar para cuales de los siguientes problemas la distribucion binomial es un modelo adecuado:
\begin{itemize}
	\item a) Determinacion de la probabilidad de que un agente de ventas lleve a cabo 2 ventas en 5 entrevistas independientes si la probabilidad es 0.25 de que el agente lleve a cabo una venta en una entrevista determinada
	\item b) Determinacion de la probabilidad de que no mas de 1 de 10 articulos producidos por una maquina sea defectuoso cuando los articulos se seleccionan a traves del tiempo y se sabe que la proporcion de defectuosos aumenta con el desgaste de la maquina con el tiempo.
\end{itemize}




\end{document}
